\section{Lenti}
Tutte le lenti obbediscono alla legge di snell sulla rifrazione, di conseguenza la geometria della lente determina come la luce si propaga attraverso gli elementi ottici, stabiliremo ora un glossario con la terminologia e introdurremo diverse geometrie di lenti.

\subsection{Terminologia}

\begin{tabularx}{\textwidth}{l p{.8\textwidth}}

D               & 	Diametro – Dimensione fisica della lente.\\
R               &	Raggio di curvatura – La distanza diretta tra il vertice di una superfice e il centro di curvatura.\\
EFL 	        &   Lunghezza focale effettiva – Distanza ottica tra il piano principale di un ottica e il piano immagine.\\
BFL 	        &   Lunghezza focale posteriore – Distanza meccanica tra l'ultima superficie della lente ed il piano immagine.\\
P, P" 	        &   Piano Principale – Piano ipotetico dove i raggi di luce incidenti si piegano a causa del fenomeno della rifrazione.\\
CT              & 	Spessore centrale – la distanza tra il piano principale e la fine dell'elemento.\\
db 	            &   Diametro di ingresso del raggio – Diametro di un raggio collimato in ingresso.\\
dr 	            &   Exit Beam Diameter – Diametro di un anello luminoso in uscita all'elemento.\\
L 	            &   Lunghezza distanza effettiva fra le superfici di un elemento.
\end{tabularx}

\subsection{Plano Convex}
Ideale per la collimazione o focalizzazione utilizzando luce monocromatica.
\begin{figure}[!ht]
\centering

\includegraphics[width=.3\textwidth]{img/plano-convex.png}

\caption{Lente Plano-Convex}
\label{fig:ccd-blockdiagram}
\end{figure}


\subsection{Double Convex}
Ideale per l'inoltro di immagini , e per l'imaging di oggetti vicini.
\begin{figure}[!ht]
\centering

\includegraphics[width=.3\textwidth]{img/double-convex.png}

\caption{Lente Plano-Convex}
\label{fig:ccd-blockdiagram}
\end{figure}

\subsection{Plano Concave}
Composta da una superficie piatta e una superficie curva verso l'interno . Ideale per l'espansione di fasci , proiezione di luce , ed espandensione della lunghezza focale del sistema ottico .

\begin{figure}[!ht]
\centering

\includegraphics[width=.3\textwidth]{img/plano-concave.png}

\caption{Lente Plano-Convex}
\label{fig:ccd-blockdiagram}
\end{figure}


\subsection{Double Concave}
Composto da due superfici curve equamente verso l'interno. Ideale per l'espansione del fascio , proiezione di luce , ed espandendo la lunghezza focale del sistema ottico .
\begin{figure}[!ht]
\centering

\includegraphics[width=.3\textwidth]{img/double-concave.png}

\caption{Lente Plano-Convex}
\label{fig:ccd-blockdiagram}
\end{figure}

\subsection{Acromatica positiva}
Esegue funzione simile a quella di una lente PCX o DCX, ma è in grado di fornire dimensioni di punto più piccole e immagini di qualità superiore . Lenti acromatiche sono utili per ridurre l'aberrazione sferica e cromatica
\begin{figure}[!ht]
\centering

\includegraphics[width=.3\textwidth]{img/positiva-acromatica.png}

\caption{Lente Plano-Convex}
\label{fig:ccd-blockdiagram}
\end{figure}

\subsection{Asferica}
Ideale per la focalizzazione laser o per la sostituzione più lenti sferiche in un sistema . Utile per eliminare l'aberrazione sferica  riducendo notevolmente le altre aberrazioni .

\begin{figure}[!ht]
\centering

\includegraphics[width=.3\textwidth]{img/asferica.png}

\caption{Lente Plano-Convex}
\label{fig:ccd-blockdiagram}
\end{figure}



