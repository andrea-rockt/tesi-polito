\section{Illuminazione}
\'{E} ampiamente riconosciuto che l'appropriatezza dell'illuminazione e la qualità della stessa
siano aspetti critici nella creazione di un sistema di visione pronto e robusto.
Progettare un ambiente di analisi robusto massimizzerà la riuscita del progetto in termini
di tempo, sforzo e risorse impiegate.

Storicamente, la luce è stato sempre l'ultimo aspetto ad essere specificato, sviluppato o finaziato, questo tipo di approccio derivava principalmente dall'assenza di prodotti commerciali espressamente rivolti alla machine vision, ciò portava all'adozioni di prodotti consumer quali lampade a incandescenza e/o fluorescenza.

Ciò che è realmente richiesto per la realizzazione di sistemi di ispezione industriale è il controllo dell' illuminazione volto a produrre:

\begin{itemize}
	\item Illuminazione appropriata dei campioni da analizzare;
	\item Standardizzazione delle componenti, tecniche, implementazioni e dell'utilizzo del sistema di illuminazione;
	\item Riproducibilità dei risultati delle ispezioni;
	\item Robustezza delle ispezioni a variazioni dell'ambiente di ispezione;
\end{itemize} 

per ottenere risultati consistenti nel design di un sistema di illuminazione bisognerebbe tenere a mente i punti di seguito riassunti:

\begin{itemize}
	\item Tipologie di luce e vantaggi e svantaggi nell'applicazione;
	\item Efficienza quantica del sensore e range spettrale;
	\item Tecniche di illuminazione e campi di applicazione relativi alla tipologia di superficie da ispezionare;
	\item Requisiti e limitazioni di ciascuna tecnica di illuminazione;
	\item Geometria della luce;
	\item Struttura (Pattern) della luce;
	\item Lunghezza d'onda della luce;
	\item Filtraggio della luce;
	\item Analisi dettagliata dell'ambiente da illuminare (requisiti e vincoli);
	\item Raccolta di campioni e verifica delle interazioni con le varie tecniche di illuminazione;
\end{itemize}

\subsection{Illuminazione direzionale}

Un illuminatore direzionale è costituito da una o più 
sorgenti di luce puntiforme che proiettano luce direzionale 
sulla parte da ispezionare, utilizzando questa tipologia di 
illuminatore è possibile ispezionare superfici piane non 
riflettenti poichè la luce raggiunge il sensore in maniera 
consistente. 

\subsection{Illuminazione tangenziale}

Un illuminatore tangenziale è costituito da una o più 
sorgente di luce direzionale aventi un elevato angolo di 
incidenza rispetto alla parte da ispezionare, ciò li rende 
adatti ad evidenziare difetti superficiali dell’oggetto che 
appaiono evidenziati sull’immagine.  
Tale metodologia e applicata con successo all’ispezione di 
componenti marcati con tecnologia DPM ( Direct Part 
Marking ) laser poichè la superficie incisa risulta evidenziata 
da questo tipo di illuminatore 

\subsection{Illuminazione diffusa}
Un illuminatore tangenziale è costituito da sorgente di luce 
diffusa ed estesa ciò li rende adatti nell’ispezione di parti 
che potrebbero creare riflessi illuminando in maniera 
omogenea e consistente l’area di ispezione, sono tuttav ia 
di difficile impiego in contesti dove gli ingombri sono 
ridotti per via delle grandi dimensioni ( Una sorgente di 
luce diffusa viene realizzata posizionando sorgenti di luce 
puntiforme lontano dall’area di ispezione )  
 
\subsection{illuminazione anulare}
Un illuminatore anulare è costituito da più sorgenti di luce 
puntiformi disposte coassialmente al dispositivo di 
imaging, ciò li rende adatti ad un ispezione simile a quelle 
possibili per mezzo di luce diffusa ma ne limita l’area 
ispezionata e la distanza di esercizio, tale tipologia di 
illuminazione produce inoltre fastidiosi riflessi circolari 
(rumore) 
 
\subsection{illuminazione diffusa assiale}
Un illuminatore diffuso assiale  è costituito da una 
sorgente di luce puntiforme direzionale orientata 
perpendicolarmente all’oggetto da ispezionare, tale luce 
colpisce un beam splitter riflettendosi prima sulla parte da 
ispezionare e poi sul dispositivo di imagin, ciò crea una 
luce diffusa senza riflessi circolari ma gli  ingombri di 
questi sistemi e la distanza di esercizio limitata ne 
vincolano l’utilizzo 

 