\chapter{Computer Vision}

In questo capitolo vorrei fare un discorso generale riguardo la computer vision,
mi piacerebbe dare un taglio cognitivo al discorso ovvero vorrei discutere di come
l'ispezione ottica richieda strumenti matematici e informatici volti ad emulare
la comprensione, mi piacerebbe anche evidenziare come la telecamera sia da considereare
un sensore tramite cui è apprezzata la realtà.

\section{Ispezione Manuale vs Ispezione automatica}

qui vorrei parlare delle esigenze risolte dalla visione computerizzata in caso di
ispezioni ottiche, quali sono gli obiettivi e le sfide, ripetibilità, oggettivizzazione,
accuratezza.

\section{La luce}

questa sezione dovrebbe descrivere con carattere non troppo tecnico l'importanza della luce
e gli effetti che può avere sull'immagine impressa sul sensore.

\section{Sensori}

questa sezione dovrebbe descrivere le varie tecnologie disponibili per la realizzazione di
sensori ottici utilizzati nelle odierne telecamere per applicazioni industriali.

\section{Algoritmi}
Qui vorrei discutere di alcuni algoritmi e tecniche importanti per il processamento delle immagini
e fare un parallelo con l'analisi dei segnali.

\subsection{Image Processing}
Qui vorrei parlare di tecniche generiche per il processamento delle immagini quali fitltraggio, filtraggio
morfologico, binarizzazione ecc.

\subsection{Object detection and recognition}	
Qui vorrei parlare di tecniche per object detection e recognition introducendo machine learning.