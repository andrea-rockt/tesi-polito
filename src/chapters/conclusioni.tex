\chapter{Conclusioni} 

Molti dei problemi odierni nell'applicazione dei sistemi
AOI al controllo qualità deriva dalla limitata intelligenza e flessibilità
degli stessi, un uomo è in grado di prendere una parte, esaminarla da diverse
angolazioni e condizioni di luce, elaborare le informazioni acquisite e quindi
esporre un proprio giudizio basato sulla propria conoscenza dell'oggetto
ispezionato o del materiale con cui è fabbricato: ciò da un punto di vista
computazionale è da paragonare all'intelligenza artificiale.

Un sistema AOI dipende attualmente da una costruzione artificiale di un \emph{palcoscenico} su
cui la parte da ispezionare viene presentata, non è in grado di comprendere
intimamente la parte ispezionata o il fine dell'ispezione.

Il miglioramento delle tecniche di
illuminazione, della capacità computazionale e del software di visione ha reso
sempre più intelligenti e flessibile l'ispezione ottica automatica ma sempre
lontana dall'intelligenza visiva umana.

Funzionalità AOI che nel passato
richiedevano hardware specializzato sono adesso realizzabili con elaboratori
general purpose. L'ispezione ottica automatica presenta molte applicazioni ma
è attualmente limitata all'individuazione di difetti visivi su parti
specificate in maniera precisa ed in condizioni ben definite, possibile
sviluppo futuro sarà la realizzazione di sistemi sempre più flessibili
coadiuvati da tecniche di machine learning, ciò porterebbe alla nascita di
stazioni di test aventi la facilità di apprendimento di un uomo ma la
velocità, l'accuratezza e la risoluzione di un elaboratore.

Attualmente grazie alla standardizzazione di alcune tecniche di illuminazione
e di posizionamento dell'oggetto si osserva la nascita sempre più veloce di
pacchetti chiavi in mano privi di programmazione che possono contenere in
maniera significativa i tempi di sviluppo di applicazioni custom.

Dopo l'anno 2000 con l'aumentare dell'integrazione dell'hardware e della
conseguente riduzione degli spazi, vediamo affermarsi la categoria di sistemi
di visione compatti chiamati smart-camera. Fondamentalmente non c'è nessuna
variazione qualitativa rispetto ai sistemi basati su PC, ma solo un aumento di
praticità. Dopo l'anno 2005 compaiono i primi sistemi integrati di visione che
sfruttano a pieno la sempre maggiore potenza dei personal computer utilizzando
software ad alto livello che usa logiche ibride. Questi sistemi di visione
hanno, diversamente dai sistemi classici, un alto livello di adattabilità agli
eventi esterni che li rendono decisamente più affidabili e versatili. Questi
sistemi sono decisamente più semplici da usare perché risolvono
automaticamente alcune problematiche tipiche della visione artificiale,
lasciando all'operatore l'incombenza di configurare solo gli aspetti
funzionali del sistema.

Durante il percorso di tesi sono stati esplorati diversi e complementari
aspetti della visione industriale dal momento che la realizzazione di sistemi
\emph{AOI} è un attività che coinvolge ottica, automazione, processamento di
immagini e apprendimento automatico.

Obiettivo di questo lavoro è stato per tanto proporre non solo software per
la visione industriale ma anche sensori, interfacce di comunicazione e
tecniche di illuminazione.

Nella progettazione e realizzazione della libreria Spea.Vision si è seguito un metodo basato 
su quanto proposto, analisi dei requisiti, definizione dell'architettura orientata agli oggetti, implementazione delle funzionalità e verifica tramite feedback dall'applicazione ai sistemi di collaudo SPEA.

Possibili sviluppi futuri dell'attività portata a termine durante il percorso
di tesi potranno essere l'arricchimento degli algoritmi di processamento di
immagini a disposizione dei tecnici SPEA nella realizzazione di programmi di
verifica ottica, una ulteriore riduzione del tempo necessario per lo
sviluppo, un miglioramento delle prestazioni de sistemi dotati di stazioni di
ispezione ottica e la realizzazione di tecniche di calibrazione ottica
necessarie a ottenere le precisioni richieste dai sistemi di ispezione e
collaudo per l'industria elettronica.



\endinput