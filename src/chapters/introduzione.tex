\chapter{Introduzione}

Tradizionalmente, l'ispezione visuale ed il controllo qualità sono stati
affidati a personale esperto[10].
Nonostante un uomo possa fare questo lavoro meglio di una macchina,
l'individuo è più lento e si stanca facilmente, oltretutto personale esperto è
difficile da trovare e tenere aggiornato in un ambito industriale richiedendo
formazione e un considerevole tempo di apprendimento; in alcuni casi
l'ispezione può essere tediosa o difficile, anche per i migliori esperti.

Alcuni anni fa, l’industria di produzione di elettronica vedeva l’introduzione
di una tecnologia di ispezione ottica automatica basata sul riconoscimento per
paragone di immagine (pattern matching). Fino ad allora, l’obiettivo
dell’ingegneria di processo era di automatizzare e velocizzare il rilevamento
dei difetti di funzionamento delle schede prodotte.Il limite di questo tipo
d’ispezione era legato alla raccolta ed all’interpretazione dei dati a
posteriori.L’evoluzione della tecnologia e l’aumento della potenza di calcolo
dei computer consente oggi diavere acquisizioni ed analisi delle immagini
estremamente rapide, permettendo ai sistemi d’ispezione ottica automatica
(AOI) di lavorare su logiche d’ispezione. Questa diversa filosofia  comporta
da un lato un’analisi quantitativa del prodotto ispezionato e dall’altra una
elevata velocità di elaborazione delle informazioni acquisite.La capacità di
misurare ed archiviare i valori durante l’ispezione consente di creare degli
storici statistici e migliorare la tracciabilità; in generale l’AOI
costituisce uno strumento per un’analisi oggettiva del processo
produttivo.Partendo da questo concetto, risulta così riduttivo considerare
l’AOI un sistema in grado di intercettare i difetti sulla scheda. Il sistema
deve essenzialmente trovare i difetti di processo, andando ad evidenziare
quando la qualità del processo stia deviando dalle condizioni iniziali.
Esistono sistemi di monitoraggio in tempo reale che sono in grado di fermare
la linea oppure di lanciare un segnale di warning a fronte di uno o più eventi
concomitanti: tramite questi è possibile intervenire prima che la deriva di
processo diventi drastica e produca una difettosità di funzionamento sulla
scheda. Questo costituisce un feedback immediato per l’operatore di
linea.L’analisi a posteriori dei dati raccolti permetterà poi al tecnologo di
evidenziare i punti deboli della linea produttiva potendo pianificare le
eventuali azioni correttive

\section{Storia}

Sebbene esistano precedenti studi e lavori, è non prima del 1970 che gli studi
nel settore si sono potuti specializzare, grazie all'aumento delle prestazioni
dei computer finalmente in grado di elaborare grandi quantità di dati dati
quali le immagini.

Negli anni 80 nascono le prime applicazioni pratiche caratterizzate spesso da uno scopo puramente dimostrativo.  

Negli anni novanta vediamo comparire i primi frame-grabber standard da
inserire su PC e i sistemi di visione acquistano maggiore funzionalità e
robustezza abbandonando l'aspetto tipicamente sperimentale del decennio
precedente, soprattutto in campo industriale si notano notevoli alti e bassi
di questa disciplina caratterizzati da alcune soluzioni funzionali costellati
di parecchi insuccessi.

Nel 2000-2008 il campo della visione artificiale può essere descritto come
vario ed immaturo, la causa va probabilmente ricercata nella sua evoluzione, a
cui hanno contribuito diverse discipline scientifiche senza però convenire su
una formulazione standard del \"problema della visione artificiale\".

Attualmente  non esiste una formulazione standard di come i problemi di
visione artificiale vadano risolti. Esistono invece un'abbondanza di metodi
atti a risolvere compiti ben definiti, dove le procedure sono spesso
dipendenti dal contesto e raramente possono essere estese ad uno spettro più
ampio di applicazioni. Molti di questi metodi sono ancora a livello di ricerca
base, ma molti altri ancora hanno trovato spazio nella produzione commerciale
dove fanno parte di grandi sistemi che risolvono problemi complessi. Nelle
applicazioni più pratiche i computer sono pre- addestrati per risolvere un
particolare compito, tuttavia diventano  sempre più comuni i metodi basati
sull'apprendimento.

Dal 2009 vediamo affermarsi l'uso delle telecamere con comunicazione digitale,
e standard che uniscono fattori come una discreta velocità, l'economicità, la
standardizzazione ed una discreta robustezza in campo industriale. In generale
l'affidabilità delle soluzioni migliora come la disponibilità di prodotti COTS
(commercial of the shelf) adatti ad un utilizzo in ambito industriale aumenta.
\endinput
