\chapter{Introduzione}

L’ Automated optical inspection (AOI) è una metodologia di ispezione visiva applicabile ad un vasto range di prodotti quali: circuiti stampati (PCB), schermi, semiconduttori, etichette su contenitori, cibi. 
Nel passato, quando un nuovo prodotto veniva realizzato, della forza lavoro veniva destinata all’ispezione visiva, la mansione di tale personale consisteva nel ricercare manualmente i difetti e assicurarsi che tutte le parti fossero posizionate correttamente sul prodotto, nonostante questo approccio fosse sufficiente, tale metodologia risultava decisamente lenta, inaccurata e non si prestava all’acquisizione automatica di informazioni su come migliorare il proprio prodotto (l’esperienza veniva acquisita dall’addetto all’ispezione ottica ma difficilmente comunicata per migliorare il processo produttivo).L’ispezione automatica, come già accennato, è disponibile per la varietà di prodotti più disparata, per esempio nell’ispezionare un frutto,un sistema AOI dovrebbe verificare variazioni di colore e/o bozzi, nell’ispezionare una parte di un autoveicolo esso dovrebbe verificare che la parte sia di dimensione corretta e che sia priva di difetti di stampaggio.

\section{Storia}

Sebbene esistano precedenti studi e lavori, è non prima del 1970 che gli studi nel settore si sono potuti specializzare, grazie all'aumento delle prestazioni dei computer finalmente in grado di elaborare grandi quantità di dati dati quali le immagini. Dobbiamo aspettare il 1980 per vedere le prime vere e proprie applicazioni pratiche di questa disciplina, caratterizzate spesso da uno scopo puramente dimostrativo. negli anni novanta vediamo comparire i primi frame-grabber standard da inserire su PC e i sistemi di visione acquistano maggiore funzionalità e robustezza abbandonando l'aspetto tipicamente sperimentale del decennio precedente, soprattutto in campo industriale si notano notevoli alti e bassi di questa disciplina caratterizzati da alcune soluzioni funzionali costellati di parecchi insuccessi. Nel 2000-2008 il campo della visione artificiale può essere descritto come vario ed immaturo. La causa va probabilmente ricercata nella sua evoluzione, a cui hanno contribuito diverse discipline scientifiche senza però convenire su una formulazione standard del \"problema della visione artificiale\". Inoltre, con conseguenze ancor più evidenti, non esiste una formulazione standard di come i problemi di visione artificiale vadano risolti. Esistono invece un'abbondanza di metodi atti a risolvere compiti ben definiti della visione artificiale, dove le procedure sono spesso dipendenti dal contesto e raramente possono essere estese ad uno spettro più ampio di applicazioni. Molti di questi metodi sono ancora a livello di ricerca base, ma molti altri ancora hanno trovato spazio nella produzione commerciale dove fanno parte di grandi sistemi che risolvono problemi complessi. Nelle applicazioni più pratiche i computer sono pre-addestrati per risolvere un particolare compito, tuttavia attualmente stanno diventando sempre più comuni i metodi basati sull'apprendimento. Dal 2009 vediamo affermarsi l'uso delle telecamere con comunicazione digitale, e standard che uniscono fattori come una discreta velocità, l'economicità, la standardizzazione ed una discreta robustezza in campo industriale. In generale l'affidabilità delle soluzioni migliora.
\endinput