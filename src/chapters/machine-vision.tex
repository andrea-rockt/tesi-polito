\chapter{Visione artificiale}

Computer vision is a field that includes methods for acquiring, processing, analyzing, and understanding images and, in general, high-dimensional data from the real world in order to produce numerical or symbolic information, e.g., in the forms of decisions.[1][2][3][4] A theme in the development of this field has been to duplicate the abilities of human vision by electronically perceiving and understanding an image.[5] This image understanding can be seen as the disentangling of symbolic information from image data using models constructed with the aid of geometry, physics, statistics, and learning theory.[6] Computer vision has also been described as the enterprise of automating and integrating a wide range of processes and representations for vision perception.[7][8]

As a scientific discipline, computer vision is concerned with the theory behind artificial systems that extract information from images. The image data can take many forms, such as video sequences, views from multiple cameras, or multi-dimensional data from a medical scanner. As a technological discipline, computer vision seeks to apply its theories and models to the construction of computer vision systems.

Sub-domains of computer vision include scene reconstruction, event detection, video tracking, object recognition, learning, indexing, motion estimation, and image restoration.

\endinput