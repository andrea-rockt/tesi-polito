\chapter{Spea.Vision Library}
Il lavoro svolto durante il percorso portato avanti in SPEA si è focalizzato
principalmente sullo sviluppo di una libreria di visione che potesse
soddisfare le esigenze di un attività industriale la cui base è l'integrazione
di unità funzionali per la realizzazione di sistemi complessi, per tanto nella sezione \ref{sec:dominio-applicativo} verrà fornita una panoramica sul dominio applicativo per poi passare nella sezione \ref{sec:requisiti} dove verranno 
descritti i requisiti raccolti durante la realizzazione e l'affinamento della libreria di visione.

\section{Dominio applicativo}
\label{sec:dominio-applicativo}

\begin{quote}
\small
In ingegneria del software e in altre discipline informatiche, l'espressione dominio applicativo (o dominio dell'applicazione; in alcuni casi dominio del problema) si riferisce al contesto in cui una applicazione software opera, soprattutto con riferimento alla natura e al significato delle informazioni che devono essere manipolate. Nei più diffusi modelli e metodi di sviluppo del software, l'analisi del dominio (ovvero l'analisi volta a comprendere il contesto operativo in cui l'applicazione dovrà inserirsi) è una componente essenziale (e in genere preliminare) e dell'analisi dei requisiti.\cite{requisiti}
\end{quote}
Una libreria di visione artificiale orientata all'utilizzo industriale comprende moduli appartenenti a diversi ambiti tecnologici, a cavallo tra hardware e software. Possiamo quindi procedere nello scomporre il dominio applicativo in differenti macro aree:

\paragraph{Acquisizione:}
Astrazione di dispositivi di acquisizione eterogenei;
\paragraph{Filtri:}
Filtri di miglioramento dell'immagine;
\paragraph{Test ottici:}
Algoritmi compositi per la realizzazione di task ricorrenti nell'ispezione ottica automatica;
\paragraph{Persistenza:}
Gestione dei formati di memorizzazione di video/immagini;
\paragraph{Conversione:}
Conversione tra spazi di colore, diversi formati di pixel;
\paragraph{Gui:}
Controlli per l'interazione uomo macchina e la visualizzazione dei risultati degli algoritmi di visione;



\subsection{Acquisizione}
In un contesto industriale differenti dispositivi possono essere impiegati per acquisire immagini quali: telecamere matriciali, lineari
o dispositivi di \emph{imaging} 3D; Differenti protocolli di comunicazione sono disponibili sul mercato; Spesso i dispositivi
espongono interfacce eterogenee, difficili da generalizzare in un unica logica di utilizzo oppure espongono funzionalità peculiari;Altrettanto spesso i dispositivi di acquisizione sono accoppiati con un \emph{triggering} fornito da dispositivi HW per tanto l'utilizzo di tali sensori richiede l'interazione con altri moduli.

\subsection{Filtri}
Il miglioramento dell'immagine iniziale porta ad un vantaggio nella successiva fase di elaborazione, spesso le immagini acquisite da sistemi industriali presentano rumore o effetti non voluti che possono essere attenuati da un accurato filtraggio. 
L'utilizzo combinato di tecniche di acquisizioni sofisticate (HDR, aumento della gamma dinamica) con i filtri può portare ad un miglioramento del rapporto segnale rumore. I filtri possono essere utilizzati anche per evidenziare determinate caratteristiche salienti nell'immagine al fine di semplificare un ispezione ottica umana.  

\subsection{Test ottici}
Alcune categorie di ispezione ottica possono essere generalizzate in degli approcci chiavi in mano basati su differenti idee di fondo. Per esempio, la localizzazione di un punto di calibrazione può avvenire per mezzo di approcci basati su ``pattern matching''; il riconoscimento di oggetti colorati può avvenire per mezzo di misure valutate in opportuni spazi di colore;Una libreria di visione industriale per tanto conterrà quanti più algoritmi di semplice utilizzo e relativa potenza.

\subsection{Persistenza}
La persistenza dei dati su disco è un altro aspetto importante della visione industriale, spesso i dati registrati dai sistemi di ispezione ottica devono essere integrati in database per il computer integrated manifacturing (CIM) e per tanto diversi formati di input output vanno gestiti e spesso essi vanno corredati di dati complementari alla sola immagine. Sono proprio i metadati a documentare e oggettivare un ispezione ottica ripetibile, per tanto le immagini vengono spesso corredate dai parametri utilizzati per l'acquisizione o dagli esiti delle fasi del test.

\subsection{Conversione}
Differenti dispositivi producono dati in formati differenti o  in spazi di colore alternativi al comune spazio RGB, per tanto è necessario un modulo in grado di tradurre le rappresentazioni in ingresso in una rappresentazione omogenea per i processi che avverranno a valle della conversione. Oltre che differenti spazi di colore anche differenti codifiche dei dati dovranno essere gestite quali codifiche sotto-campionate su alcuni canali o con differente disposizione in memoria.

\subsection{Gui}
L'interazione uomo macchina è fondamentale nell'applicazione all'industria, i risultati presentati dai sistemi di ispezione devono essere chiari e soprattutto devono catturare l'andamento statistico e le eventuali derive dei processi produttivi, per tanto componenti di visualizzazione sia di immagini che di dati statistici sono necessari alla realizzazione di sistemi efficienti.


\section{Requisiti}
\label{sec:requisiti}

\begin{quote}
\small
In ingegneria del software, l'analisi dei requisiti (talvolta detta semplicemente analisi) è un'attività preliminare allo sviluppo (o alla modifica) di un sistema software, il cui scopo è quello di definire le funzionalità che il nuovo prodotto (o il prodotto modificato) deve offrire, ovvero i requisiti che devono essere soddisfatti dal software sviluppato.
\end{quote}

Procediamo in questa sezione all'analisi dei requisiti per ogni macro area precedentemente individuata nel dominio applicativo.

\subsection{Acquisizione}
La gestione dei dispositivi di acquisizione rappresenta uno dei compiti più critici di un software di visione industriale poichè
questo modulo deve accomodare quanto più possibile la sostituzione di un dispositivo con uno di capacità equivalenti, risulta quindi necessario esplorare i seguenti requisiti.

\subsubsection{Semplice sostituzione dei dispositivi di acquisizione}
La gestione dei dispositivi di acquisizione deve avvenire in maniera generica, telecamere di produttori diversi ma di medesima natura devono essere esposte e gestite dalla libreria di visione in maniera omogenee astraendo il programmatore da eventuali peculiarità di configurazione della telecamera.
\subsubsection{Scoperta dinamica dei dispositivi di acquisizione}
Spesso un sistema di visione industriale viene aggiornato per adeguarlo ad una nuova applicazione o lo stesso sistema può lavorare con un numero differente di siti di acquisizione immagine, è opportuno quindi prevedere un meccanismo di scoperta automatico dei dispositivi di acquisizione che possa interrogare le varie interfacce supportate per la presenza di dispositivi compatibili e che possa associare gli stessi alla funzione e/o alla posizione all'interno del sistema
\subsubsection{Gestione generica dei parametri dei dispositivi}
Spesso produttori diversi espongono funzionalità e/o parametri simili in maniera leggermente diversa per tanto la libreria di visione dovrà stabilire una semantica consistente di alcune parole chiave del settore, per esempio ``tempo di esposizione'' e adottarle in maniera consistente gestendo parametri simili in maniera omogenea.
\subsubsection{Gestione del trasferimento dati}
Ogni dispositivo trasferisce i dati secondo logiche diverse, spesso si devono prevedere delle trame di esecuzione indipendenti per ogni dispositivo che servono delle code di frames è per tanto opportuno che la libreria incapsuli automaticamente le logiche di acquisizione e che gestisca correttamente, in maniera veloce ed affidabile, le politiche id acquisizione necessarie.
\subsubsection{Organizzazione in driver}
Ogni categoria di dispositivo deve essere gestita da un sottosistema omogeneo, basato sullo stesso SKD di acquisizione, in grado di distinguere le peculiarità di modelli diversi pilotati tramite lo stesso driver e di uniformarne il comportamento al resto dei driver.
\subsubsection{Gestione di segnali HW esterni}
Spesso i dispositivi di acquisizione presentano la possibilità di essere triggerati da dispositivi esterni o di pilotare sistemi di illuminazione sincronizzata tramite delle linee hw, queste funzionalità devono essere gestite in maniera omogenea per dispositivi differenti.

\subsection{Filtri}
La gestione dei filtri di acquisizione permette di migliorare la catena di acquisizione riducendo il rumore o esaltando caratteristiche salienti, nascono quindi i seguenti requisiti

\subsubsection{Gestione di catene di filtri}
I filtri devono essere liberamente componibili in catene, un approccio più evoluto e organizzato sotto forma di grafo porta alla realizzazione di veri e propri sistemi generici e grafici di connessione di filtri, per tanto i filtri devono esporre un interfaccia generica e comune standardizzando la publicazione dei metadati riguardo i filtri

\subsubsection{Filtraggio in place}
Quando vengono gestite immagini contenenti un elevato numero di dati la memoria rappresenta un reale collo di bottiglia per tanto molti filtri possono essere progettati per lavorare non su una copia dell'immagine ma sullo stesso spazio di memoria in input, i filtri devono essere in grado di publicare la possibilità di operare in-place ed il client deve essere in grado di richiederne questa particolare modalità.

\subsubsection{Passthrough}
I filtri devono essere liberamente disabilitabili in una catena, in questo caso il filtro passa semplicemente il frame corrente a valle.

\subsubsection{Formato di interscambio}
Le catene di filtri devono essere serializabili e deserializabili su disco per poter essere caricate su qualunque sw realizzato tramite l'uso della libreria.

\subsection{Test ottici}
La gestione di test ottici standard prevede la definizione di interfacce comuni per differenti esigenze quali diagnosica, metodo di esecuzione, logging dei dati

\subsubsection{Logging}
Un test ottico deve fornire dei dati di diagnostica sull'esecuzione in modo da informare il programmatore su eventuali criticità riscontrate durante l'esecuzione dell'algoritmo o per fornire dati necessari ad interpretare i risultati, per tanto un interfaccia per il logging dei dati da esporre al client deve essere realizzata.

\subsubsection{Diagnostica}
Un test ottico deve fornire oltre che a un log testuali delle immagini di diagnostica che permettano di osservare i passi intermedi di un algoritmo, la generazione di tale output deve essere disabilitabile per non inficiare i tempi di esecuzione dei test.

\subsubsection{Metodi}
Un test ottico deve esporre la possibilità di essere interrogato sui metodi supportati in modo da offrire la possibilità di eseguire algoritmi diversi a seconda dell'esigenza o deve prevedere degli opportuni punti di estensione per l'inserimento di comportamenti definiti dall'utente.


\subsection{Persistenza}
La persistena dei dati su disco rappresenta un modulo importante per l'interscambio di informazioni tra software diversi, per tanto si rilevano i seguenti requisiti.

\subsubsection{Gestione uniforme di formati immagine diversi}
Formati immagini diversi quali,jpeg, png, bmp devono essere gestiti in modo trasparente dal modulo di persistenza, eventuali peculiarità dei formati dati devono essere gestite dalla libreria e non esposti al programmatore

\subsubsection{Gestione metadati}
Metadati quali timestamp, tempo di esposizione, dispositivo di acquisizione o eventuali risultati dei test devono essere salvati e caricati contestualmente ai dati immagine

\subsubsection{Decorazione}
I meccanismi di persistenza devono supportare la decorazione ovvero l'organizzazione concentrica delle fasi di scrittura/lettura tale da rendere possibile l'estenzione del comportamento del sistema di persistenza.

\subsection{Conversione}
Il modulo di conversione rappresenta il punto di convergenza dei dati immagine verso il formato preferenziale da sottoporre agli algoritmi di filtraggio e di test ottico per tanto si individuano i seguenti requisiti.

\subsubsection{Conversione tra spazi di colore}
Deve essere possibile convertire i dati di un immagine tra rappresentazioni basate su spazi di colore diversi, deve essere possibile interrogare il sistema di conversione per ottenere un componente in grado di effettuare la conversione specificata

\subsubsection{Conversione fra differenti formati di pixel}
Deve essere possibile gestire dati immagine con profondità di colore diverse o con un diverso layout in memoria in modo da gestire formati quali YUV422 che prevedono un sottocampionamento di alcune componenti o altre peculiarità.

\subsection{Gui}
L'interazione uomo macchina deve essere gestita esponendo delle componenti grafiche semplici ed intuitive per tanto si evidenziano i seguenti requisiti.

\subsubsection{Display}
\'{E} necessario realizzare un componente grafico in grado di gestire la visualizzazione del feed proveniente da una telecamera, tale componente deve essere genericamente in grado di ricevere da una sorgente che aderisce all'interfaccia attesa e deve prevedere la possibilità di disegnare forme oppure di annotare l'immagine visualizzata con altri tipi di informazione

\subsubsection{Gamma}
I componenti grafici atti a visualizzare l'immagine devono possedere la capacità di applicare una correzione gamma alle immagini prima dell'effettiva visualizzazione in modo da rendere meglio interpretabile l'immagine visualizzata nonostante il processamento debba avvenire su una rappresentazione lineare del segnale.

\subsubsection{Interazione}
Il componente di visualizzazione delle immagini deve offrire la possibilità di effettuare  selezione, zoom, pan e di navigare fra le viste precedentemente selezionate, questo dovrà avvenire con un implementazione su GPU per non appesantire i calcoli necessari alla pipeline di visione computerizzata

\subsection{Generali}
Oltre a requisiti generati dal dominio applicativo esistono anche requisiti generati dal mercato, la maggior parte delle librerie per l'astrazione di domini applicativi legati all'applicazione industriale operano come middleware dove tecnologie diverse possono essere integrate oppure sostituite nel momento in cui rappresentano un collo di bottiglia per tanto si individuano i seguenti requisiti.

\subsubsection{Architettura basata su provider}
L'architettura della libreria di visione sarà basata sul concetto di provider, uno strato procedurale che rappresenta l'interfaccia necessaria ad adattare l'attuazione delle procedure di una libreria di image processing particolare alla dinamica del funzionamento della libreria di visione, ciò permette quindi di sostituire o aggiungere primitive provenienti da toolkit di visione differenti quali opencv o cognex vision pro utilizzandole senza preoccuparsi del marshaling dei dati tra uno e l'altro.

\section{Scelte progettuali}
\label{sec:scelte-progettuali}

Finita l'analisi dei requisiti è necessario cominciare a costruire uno stack di tecnologie solide su cui basare la realizzazione della libreria di visione, per tanto in questa sezione verranno illustrate le scelte effettuate.

\subsection{Linguaggio di programmazione}

Lo sviluppo in ambito industriale prevede bassi tempi di integrazione e di sviluppo delle applicazioni di test, ma contemporaneamente alte prestazioni di elaborazione. Le due esigenze si ritrovano fortemente in disaccordo tra loro, è proprio per questo che nella realizzazione del lavoro oggetto di tesi si è considerato di utilizzare una programmazione di tipo poliglotta, ovvero basata su differenti linguaggi di programmazione.
Il linguaggio di programmazione selezionato per la realizzazione dello strato client della libreria è il C\#, linguaggio gestito appartenente famiglia dei linguaggi aventi come target il runtime \emph{.net}. I linguaggi gestiti sono spesso evitati per la realizzazione di sistemi ad alte performance ma  possono essere utilizzati per esporre lo strato di livello più alto in modo che la programmazione sia rapida. Il linguaggio C\#  è utilizzato per la realizzazione delle componenti grafiche e dell'infrastruttura orientata agli oggetti.
Il linguaggio di programmazione selezionato per la realizzazione dello strato di processamento della libreria è il c++, grazie alla presenza delle estensioni microsoft c++/cli risulta semplice da integrare nell'infrastruttura gestita del runtime .net e la disponibilità di librerie di image processing realizzate in c++ o che forniscono dei bindings verso di esso rappresenta un vantaggio considerevole.

\subsection{Tecnologie utilizzate}

Per lo sviluppo delle componenti della libreria di visione sono state utilizzate tecnologie tra loro molto diverse fra cui

\subsubsection{WPF}
Windows Presentation Foundation (o WPF), nome in codice Avalon, è una libreria di classi del Framework .NET proprietarie Microsoft (introdotta con la versione 3.0) per lo sviluppo dell'interfaccia grafica delle applicazioni in ambienti Windows.

L'innovazione principale di WPF è la rimozione di ogni legame con il modello di sviluppo tradizionale di Windows, introdotto con la versione 1.0 del sistema operativo. Tutti i controlli sono stati riscritti (non si appoggiano più a quelli della libreria “user”) e lo stesso meccanismo basato su scambio di messaggi, cuore del modello di programmazione di Windows, è stato abbandonato.

WPF è basato su un sistema di grafica vettoriale che si appoggia alle DirectX per sfruttare l'accelerazione hardware delle moderne schede grafiche. WPF può essere impiegato per realizzare applicativi eseguibili anche all'interno del browser Microsoft Internet Explorer o di altri browser avanzati, purché sia presente il Framework. Il linguaggio usato per la creazione di una interfaccia utente in WPF è lo XAML (eXtensible Application Markup Language), basato su XML.

Tale toolkit grafico è stato selezionato per essere utilizzato nella realizzazione di questa libreria di visione poichè offre una gestione evoluta di tutte le problematiche di presentazione, infatti, esso offre punti di estensione per la scrittura di pixel shader da eseguire sulla scheda grafica, offre la possibilità di applicare trasformazioni geometriche arbitrarie ai controlli grafici permettendo quindi di gestire sistemi di coordinate arbitrarie (mondo, immagine) e tramite la definizione di template per i dati di presentare in maniera uniforme informazioni a corredo dell'ispezione ottica

\subsubsection{Opencv}
Il framework OpenCV nasce da una iniziativa dell’Intel, mentre lavorava
su miglioramenti delle loro CPU per applicazioni intensive, ad esempio
ray-tracing in tempo reale e proiezione 3D. Uno degli addetti della Intel,
aveva notato come in molte università, tra cui il MIT Media Lab fosse stato realizzato un framework di visione, il cui codice era passato
da studente a studente. A tal proposito si decise di iniziare, a partire da
questo codice, un framework per la computer vision ottimizzato per i processori Intel. Il
primo avvio di tale progetto, con la collaborazione di un team Intel russo,
fu nel 1999. la prima release ufficiale di OpenCV risale, invece, al 2006.
Opencv è stato selezionato per la disponibilità di un elevato numero di primitive di image processing, l'ottimizzazione degli algoritmi tramite librerie per il calcolo, la disponibilità del supporto al gpu computing e la genericità e flessibilità del modello di programmazione dallo stile modern c++.



\input{chapters/20-spea-vision-library/architettura.tex}


\section{Spea Optical Test App}
Questa sezione dovrebbe discutere dell'ambiente di programmazione
grafica associato alla libreria di visione, qui si discuterà del
modello di esecuzione, della risoluzione delle dipendenze tra blocchi
di image processing, di opportunità di parallelizzazione ecc.

\endinput