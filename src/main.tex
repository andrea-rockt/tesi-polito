\documentclass[11pt,a4paper,twoside,oldstyle,classica,italian]{toptesi}
\usepackage[utf8x]{inputenc}
\usepackage[italian]{babel}
\usepackage[T1]{fontenc}\usepackage{lmodern}
\usepackage{lipsum}

\ateneo{Politecnico di Torino}
\FacoltaDi{}
\facolta[III]{Facoltà di Ingegneria}
%\Materia{Remote sensing}

\titolo{Applicazioni della computer vision alle macchine di collaudo di dispositivi elettronici}
\corsodilaurea{Ingegneria Informatica}

\candidato{Andrea \textsc{Fonti}}

\relatore{prof.\ Matteo Sonza Reorda}% per la laurea e/o il dottorato
%\secondorelatore{dipl.~ing.~Werner von Braun}% per la laurea magistrale
\def\Candidato{Candidato}


%%%%%%% Tutore
%\tutore{ing.~Karl Von Braun}% per il dottorato
\tutoreaziendale{xxx}
\NomeTutoreAziendale{Supervisore aziendale\\SPEA}
\sedutadilaurea{\textsc{Anno~accademico} 2014-2015}% per la laurea magistrale
\logosede{logotesi.png}



\begin{document}\errorcontextlines=9% debugging
\frontespizio
\sommario
Questo sarà il sommario della toptesi

\indici

\chapter{Introduzione}

\chapter{Stato dell'arte}

\section{Software}
\subsection{Opencv}
\subsection{Cognex}
\subsection{Halcon}
\subsection{Mil}

\section{Hardware}
\subsection{GigE Vision Cameras}
\subsection{USB3 Vision Cameras}
\subsection{Camera Link Cameras}

\chapter{AOI - Automatic Optical Inspection}

\chapter{Spea.Vision Library}

\section{Spea Optical Test App}

\chapter{Caso di studio, Ispezione dei contatti su package BGA}

%\conclusioni
\end{document}
