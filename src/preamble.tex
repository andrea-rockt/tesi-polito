\documentclass[11pt,a4paper,twoside,italian]{toptesi}
\usepackage{hyperref}
\usepackage{subcaption}
\usepackage{lipsum}
\usepackage[utf8x]{inputenc}
\usepackage[italian]{babel}
\usepackage[T1]{fontenc}\usepackage{lmodern}
\usepackage{listings}
\usepackage{amsmath}
\usepackage[usenames,dvipsnames,rgb]{xcolor}
\usepackage{tabularx}

\newcommand{\degree}{\ensuremath{^\circ}}

\lstset{language=[Sharp]C,
showspaces=false,
showtabs=false,
breaklines=true,
showstringspaces=false,
breakatwhitespace=true,
escapeinside={(*@}{@*)},
commentstyle=\color{ForestGreen},
keywordstyle=\color{NavyBlue},
stringstyle=\color{Orange},
basicstyle=\small\ttfamily
}


\ateneo{Politecnico di Torino}
%\facolta[III]{Facoltà di Ingegneria}
%\Materia{Remote sensing}
\corsodilaurea{Ingegneria Informatica}
\titolo{Applicazioni della computer vision ai sistemi per il collaudo di
dispositivi elettronici}

\candidato{Andrea \textsc{Fonti}}

\relatore{prof.\ Matteo Sonza Reorda}% per la laurea e/o il dottorato
%\secondorelatore{dipl.~ing.~Werner von Braun}% per la laurea magistrale
\def\Candidato{Candidato}


%%%%%%% Tutore
%\tutore{ing.~Karl Von Braun}% per il dottorato
\tutoreaziendale{Cristiano Barla}
\NomeTutoreAziendale{Supervisore aziendale\\SPEA}
\sedutadilaurea{\textsc{Anno~accademico} 2014-2015}% per la laurea magistrale
\logosede{img/logo_blu.pdf}
\figurespagetrue 
%\setlength{\parindent}{0pt}

